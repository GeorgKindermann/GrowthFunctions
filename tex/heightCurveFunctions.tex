%pdflatex heightCurveFunctions
%bibtex heightCurveFunctions

\documentclass[a4paper,twocolumn]{article}
%\usepackage[top=35pt,bottom=35pt,left=48pt,right=46pt]{geometry}
\usepackage[top=10pt,bottom=10pt,left=10pt,right=10pt]{geometry}
%\linespread{1.9}
%\nolinenumbers
%\usepackage{lineno}
%\linenumbers

\usepackage{graphicx}
\usepackage{amsmath}
\usepackage{amsfonts}
\usepackage{amssymb}
\usepackage{placeins}
\usepackage{multirow}
\usepackage{multicol}
\usepackage{palatino}

\usepackage{natbib}
\usepackage{url}

\allowdisplaybreaks

%\DOI{xxxxx}

%\Year{2016}

\DeclareMathOperator{\atan}{atan}

\begin{document}

\title{Evaluation of growth functions for tree height modelling\\
Vergleich von H\"ohenzuwachsfunktionen}
%\title{Comparison of growth functions on observed tree height increments}

%\author[Evaluation of growth functions for tree height modelling]{Georg Erich \surname{Kindermann}$^{1,\ast}$}
\author{Georg Erich Kindermann}

%\address{$^{1}$Austrian Research and Training Centre for Forests, Natural Hazards and Landscape, Seckendorff-Gudent-Weg 8, 1131 Vienna, Austria}

%\corres{$^{\ast}$Corresponding author. E-mail: georg.kindermann@bfw.gv.at}

%\date{\rec{4 April 2016}}

\maketitle

Keywords: Growth function, Tree height, Site index curves, Growth model\\
Schlagworte: Zuwachsfunktion, Baumh\"ohe, Oberh\"ohenkurven, Wachstumsmodell

\begin{abstract}
  Several growth functions (e.g. Backman, Bertalanffy,
  Evolon, Fischer, Gomperz, Gram, Grosenbaugh, Hassell, Hill,
  Ho{\ss}feld, Hyperlogistic, Johnson, Koller, Korf, Kosun, K\"ovessi,
  Kumaraswamy, Levakovic, Logit, Maynard, Michaelis-Menten, Michailoff,
  Mitscherlich, Morgan, Nelder, Peschel, Richards, Ricker, Robertson,
  Schnute, Schumacher, Siven, Sloboda, Strand, Stannard, Terazaki,
  Thomasius, Todorovic, Yoshida, Weber, Weilbull) have been compared
  for their ability to follow observed height developments. Those with
  two coefficients might not be flexible enough to follow possible
  growth patterns. Some functions with three and especially those with
  four parameters can follow a wide range of growth patterns. The
  differences between most of the four parameter functions are
  marginal. Choosing one can depend on their behaviour to converge
  during parameterisation or the direct interpretability of the
  parameters selection. Overall the function \texttt{h = c0 * (ln(1 +
    c2 * t**c3))**c1}
  %$h = c_0 \cdot \ln^{c_1} (1 + c_2\cdot t^{c_3})$
  showed good behaviour and might be one which could be recommended.\\
\vskip1pt
\end{abstract}

\begin{center}
  Zusammenfassung
\end{center}
Es wurden einige Wachstumsfunktionen (z.B.\ Backman, Bertalanffy,
Evolon, Fischer, Gomperz, Gram, Grosenbaugh, Hassell, Hill,
Ho{\ss}feld, Hyperlogistic, Johnson, Koller, Korf, Kosun, K\"ovessi,
Kumaraswamy, Levakovic, Logit, Maynard, Michaelis-Menten, Michailoff,
Mitscherlich, Morgan, Nelder, Peschel, Richards, Ricker, Robertson,
Schnute, Schumacher, Siven, Sloboda, Strand, Stannard, Terazaki,
Thomasius, Todorovic, Yoshida, Weber, Weilbull) hinsichtlich ihrer
F\"ahigkeit, beobachtete H\"ohenwachstumsverl\"aufe zu beschreiben,
verglichen. Jene mit zwei Koeffizienten sind zu unflexibel um das
Spektrum an m\"oglichen Wachstumsverl\"aufen abzudecken. Funktionen mit
drei und insbesondere jene mit vier Koeffizienten k\"onnen ein breites
Spektrum an Wachtumsverl\"aufen abdecken. Die Unterschiede zwischen den
meisten Vierparameterfunktionen sind minimal, wodurch deren Auswahl
auf andere Kriterien, wie Konvergierfreudigkeit oder direkte
Interpretierbarkeit der Parameter, verlagert werden kann. Generell
zeigte die Funktion \texttt{h = c0 * (ln(1 + c2 * t**c3))**c1} gute
Resultate und kann zur Beschreibung von H\"ohenverl\"aufen empfohlen
werden.


\section{Introduction}

Site index curves are used to estimate the productivity of a forest at
a specific site. In the past those curves have been drawn
freehand. Later, averages for specific site and age classes have been
calculated and the resulting points have been connected. Nowadays
those curves are typically drawn by using a function. This function
should be flexible enough to reproduce the variety of observed height
growth patterns. In this article established functions are compared
as to how well they fit to observed height developments over time.

In the past many growth functions have been published and also several
comparisons of some of them have been done
\citep{Levakovic1935Bestandeshoehenkurve,peschel1938Wachstumsgesetze,todorovic1961GesetzmaessigkeitenDesWachstums,zeide1993AnalysisOfGrowthEquations,ricker1979GrowthRatesAndModels,elfving1997ConstructionOfSiteIndexEquations,zhang1997GrowthFunctions,chrobok2004DescriptiveGrowthModelOfTheHeightOfStapesInTheFetus,palahi2004SiteIndex,Khamis2005NonlinerGrowthModels,upadhyay2005SiteIndexEquations,koyo2013GeneralizedMathematicalModelForBiologicalGrowths,kutelka2015KORFitAnEfficientGrowthFunctionFittingTool,sedmak2015SigmoidGrowthFunction}.
It appears that there was a dissatisfaction with the older growth
functions or uncertainties which one to select. For the same reasons,
the motivation comes to write this article. It will give a collection
of more than 50 growth functions and show how they are related to each
other. It will show the typical root--mean--square error (RMSE), the
bias at some
age steps and if there is a trend over site index between observed
tree heights and each single growth function and also how the
parameterized growth function will predict heights over an age of
up to 800~years.

\section{Materials and methods}

\subsection{Observed tree height development}

Data-sets from \cite{guttenberg1915Hochgebirge} and stem analysis from
BFW (Austrian Research and Training Centre for Forests, Natural
Hazards and Landscape) have been used. The analysed trees
\textit{(Picea abies)} were raised in Austria. The used trees must
have a minimum age of 100 years and should not have been suppressed by
higher trees. The height--age curves of these trees are used as they
are and will not be identical with corresponding site index curves
which are typical created by smoothing many observations of dominant
or mean heights for many ages.
In total, 81~trees from Guttenberg and 95~trees from BFW
have been taken. Figure~\ref{fig:ageHeight} shows the height
development of these trees. In the following compilation the
distribution of age and height at the age of 50 years (h50) and
100~years (h100) is shown.

\begin{tabular}{llcccccc}
 &Src  & Min.& 25\,\%&  50\,\%& Mean& 75\,\%& Max.\\
 \hline
 \multirow{2}{*}{Age}& Gut& 100.0& 120.0& 140.0& 133.2& 150.0& 150.0\\
 & BFW& 100.0& 108.0& 129.0& 129.9& 143.0& 202.0\\[0.3em]
 \multirow{2}{*}{h50}& Gut& 2.70& 9.30& 12.90& 13.27& 17.50& 23.20\\
 & BFW& 5.45& 11.12& 14.84& 14.94& 19.00& 27.20\\[0.3em]
 \multirow{2}{*}{h100}&Gut& 7.70& 17.90& 23.40& 23.51& 29.10& 37.00\\
 & BFW& 10.97& 22.06& 24.74& 25.89& 31.08& 36.71
\end{tabular}

\begin{figure*}[htbp]
  \centering
  \includegraphics[height=.85\textheight]{../pic/ageHeight}
  \caption{Height development of the trees}
  \label{fig:ageHeight}
  \scriptsize{The height at age~0 is shifted from tree to tree by one meter.}
\end{figure*}


\subsection{Growth functions}

In literature many growth functions can be found. Also different
requirements, which the function should fulfil, are stated. Returning
a size of zero at time zero seems to be a good behaviour of the
function. But in most cases it might be good enough if it returns
something close to zero at time zero. Also the question arises when is
time zero. Is it zero when the egg cell is built, when it is
fertilised, when the seed is distributed, when the seed starts to
germinate, \dots? There is no need that it returns a finite number at
infinite time, no need to be asymptotic, as this time is never
reached. It is good enough when the returned sizes at ages, which could
be maximally reached, are still in a reasonable range. The conditions
that the increments and also the slope of the increments at time
zero have to be zero are questionable and could be determined by the
observations. Also a nice behaviour is if the function returns an
equal or larger value for an increased time. If it shows a
peak, the size after this peak could be kept at this peak for older
ages. Here the functions are used as they are. So no
modifications afterwards occur.  The growth function should have an
inflection point because height increment typically shows a maximum.

Growth functions can be distinguished in functions which estimate the
actual size at a certain point in time and functions which estimate
the change of size at a given time or size. Those types can be
transformed to each other by integration and differentiation
respectively. The change of size can be estimated absolutely or relatively
(e.g. change in percentages of current size). In the past many authors
have done research in this field, so only a selection of popular
functions could be made. One of the simplest equation will be the
linear function (eq.~\ref{eq:linear}) where $h$ is the tree height,
$t$ is the age and $c_0$ is a coefficient. Here the size is linear
increasing with time. It will represent the lower benchmark which
should be exceeded in most cases by all other functions. A pragmatic
extension can be the additional use of $t^2$ like in
function~\ref{eq:parabolar} which will show a maximum. This function
could be extended further with some polynomials as shown in
function~\ref{eq:poly3} and \ref{eq:poly4}. Another way to increase
the curve flexibility can be reached by powering the age with a
coefficient as shown in function~\ref{eq:allometric} which is
equivalent to $h = c_0 (1 - e^{c_1 \cdot t^{-1}} )^{c_2}$.
Polynomial and allometric can also be combined as shown in
function~\ref{eq:polyAllo}.\\
%
Another way would be to use concave monotonically increasing functions
and place exponents on possible places like shown for $\ln$ in
function~\ref{eq:ln}, for $\atan$ in function~\ref{eq:atan} and for
$\tanh$ in function~\ref{eq:tanh}.\\
%
\cite{terazaki1915GrowthCurves} introduced
function~\ref{eq:terazaki} which was reinvented e.g.\ by
\cite{johnson1935ATrendLineForGrowthSeries},
\cite{Schumacher1939Function} and \cite{michailoff1943Hoehenkurven}.
This function can be extended by adding the term $c_2\cdot
t^{-0.5}$ as shown in function~\ref{eq:terazakiE1} or by estimating
the exponent with a coefficient as shown in
function~\ref{eq:terazakiE2}.
The function of \cite{korf1939Funktion} is an extended version of
Terazakis function as the fixed exponent $-1$ is exchanged by a
coefficient (function~\ref{eq:korf}). Function~\ref{eq:gomperz} from
\cite{gompertz1825} looks again similar to
function~\ref{eq:korf}. Only the term $t^{c_2}$ is exchanged with
$e^{c_2\times t}$.
$h = \frac{c_0}{(1+c_1 \cdot e^{c_2 \cdot t})^{c_3}}$ 
was published by \cite{stannard1985GrowthFunction} which produces
quasi equivalent results like eq.\,\ref{eq:gomperz}.
\cite{sloboda1971Wachstumsprozesse}
again extended function~\ref{eq:gomperz} by introducing an exponent to
$t$ which results in function~\ref{eq:sloboda}.
\cite{grosenbaugh1965SigmoidFunction} published the function
$h = c_0 + c_1\cdot (e^{(c_2^2 - 1)\cdot U} - c_2\cdot U)^{c_2\cdot c_3 + 1}$
where $U$ is a function such as $U = e^{c_4\cdot t}$. His
function is able to describe growth like a couple of other growth
functions. This is possible by selecting specific values for some
coefficients and using different functions for the
parameter $U$. For instance by setting $c_2=0$, $c_0=0$ and using
$U = c_4\cdot e^{c_5\cdot t^{c_6}}$ it will be exactly the same like
function~\ref{eq:sloboda}. In the tested function~\ref{eq:Grosenbaugh}
the first term was eliminated, $U = e^{c_4\cdot t^{c_5}}$ was used and
the coefficient numbers have been reassigned.\\
%
Function~\ref{eq:korsun} was used by \cite{kosrun1935Funktion} which
is similar to function~\ref{eq:backman} from
\cite{backmann1931DasWachstumsproblem} with the difference that one
estimates the height, the other the height increment. This function
has been extended, as shown in function~\ref{eq:korsunE1} and
\ref{eq:korsunE2} by adding another element to the polynomial or by
exchanging the fixed exponent by a coefficient.\\
%
\cite{weber1891Forsteinrichtung} has propagated
function~\ref{eq:weber}. \cite{weber1891Forsteinrichtung} skipped the
first years and applied then his function. Here this function is used
for all ages.
\cite{puettner1920StudienUeberPhysiologischeAehnlichkeit6Wachstumsaehnlichkeiten}
and
\cite{bertalanffy1934UntersuchungenUeberDieGesetzlichkeitenDesWachstumsTeilI}
have used $h = c_0 (1 - c_2 \cdot e^{c_1 \cdot t})$, which also was
called ``monomolecular'', according to the speed of this chemical reaction.
Tests of this function ended with the result that most of the time
$c_2 = 1$ was estimated and in cases it was not 1 the function returns
a height unequal zero at age zero. This function is sometimes also
named after \cite{mitscherlich1909Bodenertrag}. There are several
ways to extend function~\ref{eq:weber}. One is to combine some of these
functions by multiplication as shown in function~\ref{eq:weberE1} and
\ref{eq:weberE2}. It is also possible to add an exponent as shown in
function~\ref{eq:weberE3}. \cite{koevessi1928ffGesetzmaessigkeiten}
joined two functions of Weber together by adding them as shown in
equation~\ref{eq:koevessi1}.
\cite{koevessi1929ffGesetzmaessigkeiten4} reduced his function to a
three coefficient form as shown in equation~\ref{eq:koevessi2}. By
adding an exponent to Webers function, function~\ref{eq:mitscherlich}
will be the result which was introduced by
\cite{mitscherlich1919Pflanzenwachstum}. If this exponent is set to
$c_2 = 3$ the function was sometimes named after ``Bertalanffy''. This
exponent $c_2 = 3$ comes from transforming observed length into
weight. \cite{schnute1981AVersatileGrowthModel} showed a function
$h= (y_1^b + (y_2^b-y_1^b) \cdot \frac{1 - e^{a*(t-t_1)}} {1 - e^{a*(t_2-t_1)}})^{(1/b)}$
which can be transferred to Mitscherlich's
function if the two selectable ages are set to $t_1=0$ and
$t_2=\infty$ and the sizes ($y$) at these ages are $y_1=0$ and $y_2$ gets the
highest possible height. \cite{richards1959GrowthCurve} added to this
function an additional factor
$h = c_0 (1 - {c_4} \cdot e^{c_1 \cdot t} )^{c_2}$ where $c_4$
sets the size at time zero depending on the the coefficients. Tests of this
function ended with the result that most of the time $c_4 = 1$ was
estimated and when it was not 1 the function returns a height
unequal zero at age zero. \cite{chapman1931Function} also used this
function which was then named the Chapman--Richards--Function.
Function~\ref{eq:weber} could also be extended by adding an exponent
to $t$ which is shown in function~\ref{eq:fischer}
\citep{fisher1928Funktion}. In
$h = c_0 (1 - {c_4} \cdot e^{c_1 \cdot t^{c_3}})$ the coefficient $c_4$
is added to eq.~\ref{eq:fischer}
which is called after Weilbull. Tests of this function ended with the
result that $c_4 = 1$ was most of the time estimated and in cases it
was not 1 the function returns a height unequal zero at age zero.
%
$h = c_0 (1 - (1 - (\frac{t}{c_3})^{c_1})^{c_2})$ was developed by
\cite{kumaraswamy1980DensityFunction} which converges to
eq.\,\ref{eq:fischer} if $c_3$ has high values. Coefficient estimates
for the used data set showed high values for $c_3$ and estimates close
to identical to those from eq.\,\ref{eq:fischer}.
\cite{todorovic1961GesetzmaessigkeitenDesWachstums} extended this
function type to four coefficients which leads to
function~\ref{eq:todorovic}. Another extension of Weber's function was
done by \cite{thomasius1964duengung} and is given in
function~\ref{eq:thomasius}.\\
%
The functions~\ref{eq:hossfeld1}, \ref{eq:hossfeld3} and
\ref{eq:hossfeld4} are from
\cite{hossfeld1822Mathematik}. Equation~\ref{eq:hossfeld1E1} shows a
simple extension of function~\ref{eq:hossfeld1} by adding the term
$c_3 \cdot t^{-3}$. The function from
\cite{morgan1975ModelForNutritionalResponse} is equivalent to
\ref{eq:hossfeld4} and could be alternatively formulated as
$h = \frac{c_0}{1+c_1 \cdot e^{c_2 \cdot \ln(t)}}$. Also the functions
by
\cite{hill1913TheCombinationOfHaemoglobinWithOxygenAndWithCarbonMonoxide}
, \cite{michaelis1913DieKinetikDerInvertinwirkung},
\cite{maynard1973TheStabilityOfPredatorPreySystems} or
\cite{hassell1975DensityDependenceInSingleSpeciesPopulations} are equal
to \ref{eq:hossfeld4}.
\cite{yoshida1928Zuwachskurve} exchanged the term
$c_2 \cdot t^{-2}$ from Ho{\ss}feld1 (eq.~\ref{eq:hossfeld1}) with $c_2
\cdot t^{c_3}$ as shown in function~\ref{eq:yoschida}.
\cite{strand1964SiteIndexCurves} has set the exponent $c_2=-1$ of
Ho{\ss}feld4 (eq.~\ref{eq:hossfeld4}) and powered the
denominator by 3 (eq.~\ref{eq:strand}).
\cite{Levakovic1935Bestandeshoehenkurve} already generalised the
function of Strand by exchanging the exponent $3$ with a coefficient
(eq.~\ref{eq:levakovic1}) and in a further step he exchanged the
exponent $-1$ also with a coefficient (eq.~\ref{eq:levakovic2}). This
function was reinvented e.g.\ by
\cite{gottschalk2005FiveParameterLogistic}.
In this function group (eq.~\ref{eq:hossfeld3} -- \ref{eq:levakovic2})
typically a
division by zero at time zero is produced, so it would sometimes help
to reform those functions.
$h = c_0 \cdot (1 - \frac{1}{1 + c_1\cdot t^{c_2}})^{c_3}$ will
produce the same like Levakovic159 (eq.~\ref{eq:levakovic2}) but
avoiding the division by zero.\\
%
The logit function is shown in eq.~\ref{eq:logit} which is identical
with the function by \cite{robertson1908OnTheNormalRateOfGrowth} and
named ``autocatalytic'' as the speed of this chemical reaction can be
described with it. The basic form of eq.~\ref{eq:logit} could be extended
by adding an exponent to age as shown in
eq.~\ref{eq:logitE1}. Alternatively also a polynomial could be used as shown
in eq.~\ref{eq:logitE2} with the untransformed age or in
eq.~\ref{eq:logitE3} with the logarithm of age. \cite{nelder1962Function}
added an exponent to the whole denominator as shown in
eq.~\ref{eq:nelder}.
\\
\cite{siven1896GrundsaetzeZurBerechnungDesHoehenwachstumsDerNadelhoelzer}
has estimated tree height with
 $h = h_x \cdot (\frac{t}{t_x})^{0.25\cdot \sqrt{t/t_x}}$ 
where $h_x$ is the height at time $t_x$. A generalised form is given
in function~\ref{eq:siven}.
\cite{gram1879KostruktionVonZuwachsuebersichten} uses
function~\ref{eq:gram} to estimate height over
age which is similar to \cite{ricker1954StockAndRecruitment}.
\cite{peschel1938Wachstumsgesetze} has formulated
function~\ref{eq:peschel} based on some predefined assumptions.

The previous functions describe the absolute size at a given time. The
following functions describe the change of size at a given time.
\cite{backmann1931DasWachstumsproblem} has used
function~\ref{eq:backman} to describe also tree height growth. For
this purpose he has used this function in a series of up to three
functions and combined their results either by adding or by
multiplying the sub-function results. Here this combination was not
made. This function was extended by adding the term $\ln^4(t)$ as
shown in equation~\ref{eq:backmanE1}. \cite{koller1886Zuwachskurven}
has used function~\ref{eq:koller} which was extended by
\cite{Levakovic1935Bestandeshoehenkurve} as shown in
equation~\ref{eq:levakovicIh}.

The hyperlogistic differential equation was formulated as
$ih = c_0 \cdot (c_1 + h)^{c_2} \cdot (c_3-(c_1 + h)^{c_4})^{c_5}$
which is based on Euler's beta function and shows the change of size
for a given size. It is based on works by
\cite{pearson1934tables},
\cite{richards1959GrowthCurve},
\cite{nelder1961GeneralizedLogisticCurv},
\cite{blumberg1968LogisticGrowthRateFunctions},
\cite{turner1969AGeneralizationOfTheLogisticLawOfGrowth},
\cite{turner1976ATheoryOfGrowth} or
\cite{buis1991LogisticGrowth}.
It can be assumed, that only
two of the three exponents are significant and so typically
$c_4=1$. This function has been used by \cite{mendeAlbrecht2001} to
describe a height curve and named the function Evolon-Model. The form
shown in function~\ref{eq:hyperlogistic} was used for coefficient
estimation.
In function~\ref{eq:hyperlogistic} the coefficient $c_1$ is needed to
allow an increment estimation different from zero when $h=0$. The same
problem can be solved with function~\ref{eq:hyperlogisticE1}. If an
upper limit is not needed or wanted, alternatively
$e^{c_3 \cdot  h^{c_4}}$ can be used instead of $(1-h/c_3)^{c_4}$
which is shown in function~\ref{eq:hyperlogisticE2} which is similar
to the formulation of Gram (eq.~\ref{eq:gram}), Koller
(eq.~\ref{eq:koller}) or Levakovic147 (eq.~\ref{eq:levakovicIh}). Also
\cite{zeide1993AnalysisOfGrowthEquations} has published similar functions.
Function~\ref{eq:hyperlogisticE3}, \ref{eq:hyperlogisticE4} and
\ref{eq:hyperlogisticE5}
show additional examples how the second term
could be exchanged with $1/(1 + c_3\cdot h^{c_4})$,
$1 - \tanh(c_3 \cdot h^{c_4})$ and
$\frac{\pi}{2} - \atan(c_3\cdot h^{c_4})$ respectively.
To those terms an additional exponent could be added like
$1/(1 + c_3\cdot h^{c_4})^{c_5}$,
$1 - \tanh^{c_5}(c_3 \cdot h^{c_4})$ and
$\frac{\pi}{2} - \atan^{c_5}(c_3\cdot h^{c_4})$. It could also be
tried to exchange the term $c_0 + c_1 \cdot h^{c_2}$ by some other
function or to combine the accelerator and decelerator by a
subtraction instead of a multiplication.

It also might be possible to use some of the functions previously shown 
to describe the increment at a specific size simply by exchanging
time with size. In the equation from
Backman (eq.~\ref{eq:backman}) and Korsun (eq.~\ref{eq:korsun}) or
Gram (eq.~\ref{eq:gram}) and Koller (eq.~\ref{eq:koller}) something
similar is done, as the same function describes once height and some
other time height increment.

\begin{table*}
\begin{minipage}[t]{.5\textwidth}
\begin{align*}
  \label{eq:linear}
  h & = c_0 \cdot t & Linear \tag{1.0}\\
  \label{eq:parabolar}
  h & = c_0 \cdot t + c_1 \cdot t^2 & Parabola \tag{1.1}\\
  \label{eq:poly3}
  h & = c_0 \cdot t + c_1 \cdot t^2 + c_2 \cdot t^3 & Polynom3 \tag{1.2}\\
  \label{eq:poly4}
  h & = c_0 \cdot t + c_1 \cdot t^2 + c_2 \cdot t^3 + c_3 \cdot t^4 & Polynom4 \tag{1.3}\\
  \label{eq:allometric}
  h & = c_0 \cdot t^{c_1} & Allometric \tag{1.4}\\
  \label{eq:polyAllo}
  h & = c_0 \cdot t + c_1 \cdot t^2 + c_2 \cdot t^{c_3} & PolyAllo
  \tag{1.5}\\[1em]
  \label{eq:atan}
  h & = c_0 \cdot \atan^{c_1} (c_2\cdot t^{c_3}) & Atan \tag{2.0}\\
  \label{eq:tanh}
  h & = c_0 \cdot \tanh^{c_1} (c_2\cdot t^{c_3}) & Tanh \tag{2.1}\\
  \label{eq:ln}
  h & = c_0 \cdot \ln^{c_1} (1 + c_2\cdot t^{c_3}) & Ln \tag{2.2}\\
  \label{eq:lnFixC1}
  h & = c_0 \cdot \ln^{c_{1,fix}} (1 + e^{c_2}\cdot t^{c_3}) & LnFixC1 \tag{2.3}\\
  \label{eq:lnFixC3}
  h & = c_0 \cdot \ln^{c_1} (1 + e^{c_2}\cdot t^{c_{3,fix}}) & LnFixC3 \tag{2.4}\\
  \label{eq:lnFixC1C3}
  h & = c_0 \cdot \ln^{c_{1,fix}} (1 + e^{c_2}\cdot t^{c_{3,fix}}) & LnFixC1C3 \tag{2.5}\\
  \label{eq:lnFixC1C3Fn}
  h & = c_0 \cdot \ln^{c_{1,fix}} (1 + e^{c_2}\cdot t^{c_{3,fix} + c_2\cdot c_{4,fix}}) & LnFixC1C3Fn \tag{2.6}\\[1em]
  \label{eq:terazaki}
  h & = c_0 \cdot e^{c_1\cdot t^{-1}} & Terazaki \tag{3.0}\\ %Logarithmic
  \label{eq:terazakiE1}
  h & = c_0 \cdot e^{c_1\cdot t^{-1} + c_2\cdot t^{-0.5}} & TerazakiE1 \tag{3.1}\\
  \label{eq:terazakiE2}
  h & = c_0 \cdot e^{c_1\cdot t^{-0.5} + c_2\cdot t^{c_3}} & TerazakiE2 \tag{3.2}\\
  \label{eq:korf}
  h & = c_0 \cdot e^{c_1\cdot t^{c_2}} & Korf \tag{3.3}\\
  \label{eq:gomperz}
  h &= c_0 \cdot e^{c_1 \cdot e^{c_2 \cdot t}} & Gompertz \tag{3.4}\\
  \label{eq:sloboda}
  h &= c_0 \cdot e^{c_1 \cdot e^{c_2 \cdot t^{c_3}}} & Sloboda \tag{3.5}\\
  \label{eq:Grosenbaugh}
  h &= c_0 \cdot (e^{(c_1^2-1) \cdot e^{c_2 \cdot t^{c_3}}} & Grosenbaugh \tag{3.6}\\
  \nonumber & \qquad - c_1 \cdot e^{c_2 \cdot t^{c_3}})^{c_1*c_4+1}\\
  \label{eq:korsun}
  h & = c_0 \cdot e^{c_1 \cdot \ln(t)+c_2 \cdot \ln^2(t)} & Korsun \tag{3.7}\\
  \label{eq:korsunE1}
  h & = c_0 \cdot e^{c_1 \cdot \ln(t)+c_2 \cdot \ln^2(t)+c_3 \cdot \ln^4(t)} & KorsunE1 \tag{3.8}\\
  \label{eq:korsunE2}
  h & = c_0 \cdot e^{c_1 \cdot \ln(t)+c_2 \cdot \ln^{c3}(t)} & KorsunE2 \tag{3.9}\\[1em]
  \label{eq:weber}
  h & = c_0 (1 - e^{c_1 \cdot t} ) & Weber \tag{4.0}\\
  \label{eq:weberE1}
  h & = c_0 (1 - e^{c_1 \cdot t} ) \cdot (1 - e^{c_2 \cdot t} ) & WeberE1 \tag{4.1}\\
  \label{eq:weberE2}
  h & = c_0 (1 - e^{c_1 \cdot t} ) \cdot (1 - e^{c_2 \cdot t} ) & WeberE2 \tag{4.2}\\
  \nonumber & \qquad \cdot (1 - e^{c_3 \cdot t} )\\
  \label{eq:weberE3}
  h & = c_0 (1 - e^{c_1 \cdot t} ) \cdot (1 - e^{c_2 \cdot t} )^{c_3} & WeberE3 \tag{4.3}\\
  \label{eq:koevessi1}
  h &= c_0 \cdot (1 - e^{c_1\cdot t}) + c_2 \cdot (1 - e^{c_3\cdot t}) & \text{K\"ovessi1} \tag{4.4}\\
  \label{eq:koevessi2}
  h &= c_0(\frac{1 - e^{-c_1\cdot t}}{c_1} - \frac{1 - e^{-c_2\cdot t}}{c_2}) & \text{K\"ovessi2} \tag{4.5}\\
%  \label{eq:bertalanffy}
%  h & = c_0 (1 - e^{c_1 \cdot t} )^{3} & Bertalanffy \tag{1.1}\\
  \label{eq:mitscherlich}
  h & = c_0 (1 - e^{c_1 \cdot t} )^{c_2} &
  Mitscherlich \tag{4.6}\\
  \label{eq:fischer}
  h & = c_0 (1 - e^{c_1 \cdot t^{c_3}} ) & Fischer \tag{4.7}\\
%%  \label{eq:XX} %gleich wie logistic
%%  h & = c_0 (1 - e^{c_1 \cdot t^{-1}} )^{c_2} & -- \tag{1.4}\\
  \label{eq:todorovic}
  h & = c_0 (1 - e^{c_1 \cdot t^{c_3}} )^{c_2} & Todorovic \tag{4.8}\\
  \label{eq:thomasius}
  h & = c_0 (1 - e^{c_1 \cdot t \cdot (1 - e ^{c_2 \cdot t})}) &
  Thomasius \tag{4.9}%\\[1em]
\end{align*}
\end{minipage}%
\begin{minipage}[t]{.5\textwidth}
\begin{align*}
%%  \label{eq:monomolecular}
%%  h &= c_0 (1-c_4 \cdot e^{c1 \cdot t}) & Monomolecular \tag{1.6}\\
%%  \label{eq:richards}
%%  h & = c_0 (1 - {c_4} \cdot e^{c_1 \cdot t} )^{c_2} & Richards \tag{1.7}\\
%%  \label{eq:weilbull}
%%  h & = c_0 (1 - {c_4} \cdot e^{c_1 \cdot t^{c_3}} ) & Weilbull \tag{1.8}\\
%%  \label{eq:XX}
%%  h & = c_0 (1 - {c_4} \cdot e^{c_1 \cdot t^{c_3}} )^{c_2} & -- \tag{1.9}\\[1.5em]
  \label{eq:hossfeld3}
  h & = \frac{c_0 \cdot t}{c_1 + c_2 \cdot \ln(t) + t} & Ho{\ss}feld 3 \tag{5.0}\\
  \label{eq:hossfeld1}
  h & = \frac{c_0}{1+c_1 \cdot t^{-1} + c_2 \cdot t^{-2}} & Ho{\ss}feld 1 \tag{5.1}\\
  \label{eq:hossfeld1E1}
  h & = \frac{c_0}{1+c_1 \cdot t^{-1} + c_2 \cdot t^{-2} + c_3 \cdot t^{-3}} & Ho{\ss}feld 1E1 \tag{5.2}\\
  \label{eq:yoschida}
  h & = \frac{c_0}{1+c_1 \cdot t^{-1} + c_2 \cdot t^{c_3}} & Yoschida \tag{5.3}\\
%%  \label{eq:stannard}  %Praktisch identisch mit Gompertz
%%  h &= \frac{c_0}{(1+c_1 \cdot e^{c_2 \cdot t})^{c_3}} & Stannard \tag{5}\\
  \label{eq:hossfeld4}
  h & = \frac{c_0}{1 + c_1 \cdot t^{c_2}} & Ho{\ss}feld4 \tag{5.4}\\
  \label{eq:strand}
  h & = \frac{c_0}{(1 + c_1 \cdot t^{-1})^{3}} & Strand \tag{5.5}\\
  \label{eq:levakovic1}
  h & = \frac{c_0}{(1 + c_1 \cdot t^{-1})^{c_3}} & Levakovic88 \tag{5.6}\\
  \label{eq:levakovic2}
  h & = \frac{c_0}{(1 + c_1 \cdot t^{c_2})^{c_3}} &
  Levakovic159 \tag{5.7}\\[1em]
  \label{eq:logit}
  h &= \frac{c_0}{1+c_1 \cdot e^{c_2 \cdot t}} & Logit \tag{6.0}\\
%  \label{eq:robertson}
%  h &= c_0 \cdot (1 - \frac{1}{1 + e^{c_1 \cdot (t - c2)}}) & Robertson \tag{55}\\
  \label{eq:logitE1}
  h &= c_4 + \frac{c_0}{1+c_1 \cdot e^{c_2 \cdot t^{c_3}}} & LogitE1 \tag{6.1}\\
  \label{eq:logitE2}
  h &= c_4 + \frac{c_0}{1+c_1 \cdot e^{c_2 \cdot t + c_3 \cdot t^2}} & LogitE2 \tag{6.2}\\
  \label{eq:logitE3}
  h &= \frac{c_0}{1+c_1 \cdot e^{c_2 \cdot \ln(t) + c_3 \cdot \ln^2(t)}} & LogitE3 \tag{6.3}\\
  \label{eq:nelder}
  h &= c_4 + \frac{c_0}{(1+c_1 \cdot e^{c_2 \cdot t})^{c_3}} & Nelder \tag{6.4}\\[1em]
  \label{eq:siven}
  h & = c_0 \cdot (t/c_1)^{c_2\cdot t^{c_3}} & Siven \tag{7.0}\\
  \label{eq:gram}
  h & = c_0 \cdot t^{c_1} \cdot e^{c_2 \cdot t} & Gram \tag{7.1}\\
  \label{eq:peschel}
%  h &= c_0 \cdot (1 - e^{-2\frac{t}{c_1}} \cdot (1 + 2\cdot \frac{t}{c_1} + 2\cdot (\frac{t}{c_1})^2)) & Peschel \tag{56}\\
  h &= c_0 \cdot (1 - e^{-2\frac{t}{c_1}} & Peschel \tag{7.2}\\
  \nonumber & \qquad \cdot (1 + 2\cdot \frac{t}{c_1} + 2\cdot
  (\frac{t}{c_1})^2))\\[1em]
  \label{eq:backman}
  ih & = c_0 \cdot e^{c_1 \cdot \ln(t)+c_2 \cdot \ln^2(t)} & Backman \tag{8.0}\\
  \label{eq:backmanE1}
  ih & = c_0 \cdot e^{c_1 \cdot \ln(t)+c_2 \cdot \ln^2(t) \cdot c_3 \ln^4(t)} & BackmanE1 \tag{8.1}\\
  \label{eq:koller}
  ih &= c_0 \cdot t^{c_1} \cdot c_2^{-t} & Koller \tag{8.2}\\
  \label{eq:levakovicIh}
  ih &= c_0 \cdot t^{c_1} \cdot c_2^{-t^{c3}} & Levakovic147 \tag{8.3}\\[1em]
  \label{eq:hyperlogistic}
  ih &= c_0 \cdot (c_1 + h)^{c_2} \cdot (1-(c_1 + h)/c_3)^{c_4} & Hyperlog \tag{9.0}\\
  \label{eq:hyperlogisticE1}
  ih &= (c_1 + c_0 \cdot h^{c_2}) \cdot (1-h/c_3)^{c_4} & HyperlogE1 \tag{9.1}\\
  \label{eq:hyperlogisticE2}
  ih &= (c_0 + c_1 \cdot h^{c_2}) \cdot e^{c_3 \cdot h^{c_4}} & HyperlogE2 \tag{9.2}\\
  \label{eq:hyperlogisticE3}
  ih &= (c_0 + c_1 \cdot h^{c_2}) \cdot (1/(1 + c_3\cdot h^{c_4})) & HyperlogE3 \tag{9.3}\\
  \label{eq:hyperlogisticE4}
  ih &= (c_0 + c_1 \cdot h^{c_2}) \cdot (1 - \tanh(c_3 \cdot h^{c_4})) & HyperlogE4 \tag{9.4}\\
  \label{eq:hyperlogisticE5}
  ih &= (c_0 + c_1 \cdot h^{c_2}) \cdot (\frac{\pi}{2} - \atan(c_3\cdot h^{c_4})) & HyperlogE5 \tag{9.5}
\end{align*}
\end{minipage}
\end{table*}

During the comparison it turned out that eq.~\ref{eq:ln} showed good
results. So this function was further examined. To facilitate the
parameter estimation $c_2$ can be exchanged with $e^{c_2}$ and the
function is formulated as $h = c_0\cdot \ln^{c_1}(1 + e^{c_2} \cdot
t^{c_3})$. In the next step it was tried to keep coefficients constant
for each of the the two data-set. In eq.~\ref{eq:lnFixC1} $c_1$ and in
eq.~\ref{eq:lnFixC3} $c_3$ was kept constant. It was also tried to
keep both coefficients $c_1$ and $c_3$ constant
(eq.~\ref{eq:lnFixC1C3}). It was also tested if this function, with
two varying parameters, can be improved by allowing a linear
dependency of $c_3$ on $c_2$ (eq.~\ref{eq:lnFixC1C3Fn}).

\subsection{Height curve parameter estimation}

The function coefficients are estimated with nonlinear least square
methods for each single tree using \cite{r2015}. Such methods need
predefined starting parameters from which their coefficient
optimisation begins. Bad starting parameters can lead to sub-optimal
results. Finding one set of starting parameters which will work for
each single tree might not be possible in each case. When it's
possible to linearize the equation, individual starting coefficients
for each tree are used. It's also possible to try a range of different
starting parameters and select those which have shown the best result. The
starting parameter search was done with the function {\tt nls2}
\citep{nls2_2013,proto2012}
%(packages nls2\_0.2 and proto\_0.3-10)
using the algorithm {\tt brute-force}. The final coefficient estimates
have been done with the function {\tt nlsLM} \citep{minpackLm2015}
with the algorithm {\tt Levenberg-Marquardt}.

Parameter estimation can be hard
if some of them are correlated or have a huge range of possible
values. To speed up or enable the algorithm to converge, the equation
can be reformulated e.g.\ by multiplying one of the coefficients with
the other and to use the coefficients in a transformed way like $C_0 =
1/c_0$ or $C_0 = exp(c_0)$.

Sometimes the data-point $t=0$, $h=0$
makes problems. In these cases this observation was not used. As most
of the functions meet per definition this point, such a data reduction
will not influence the coefficient estimates. The estimated
standard-error will be influenced and needs to be calculated using
all data. Alternatively this data point ($t=0, h=0$) could be removed
for all functions, but some functions (e.g.\ Gompertz or Logit) do not
hit this point. Functions which do not hit the data point ($t=0, h=0$)
would benefit if this data point would be removed and the relation to
other functions might be skewed.

All functions which are estimating the increment $ih$ have been
numerically integrated to a height over age curve. The starting point
was $t=0$ and $h=0$. $ih$ was not used as the slope of the tangent, it 
was used as the annual height increment. Coefficients have been
estimated for least square fit of this height over age curve
integral. With the present data (age and height) the annual height
increment could be approximately estimated and used directly in these
equations. Those coefficients are optimal to estimate, for a given age
or height, the increment of the next year, but they need not
automatically be optimal for the resulting growth curve over the whole
observation time of the tree. As $ih$ was used as the annual height
increment, those functions need to estimate $ih > 0$ for the
point where $h=0$.

\subsection{Differences between observation and model}

The RSME between the observed and the estimated
height was calculated for each tree and each growth function with
$RSME = \sqrt{\sum_{1}^{n_{obs}} (h_{obs} - h_{est})^2 / (n_{obs} - n_{coef})}$
where $n_{obs}$ is the number of height observations, $h_{obs}$ is
the observed heights, $h_{est}$ are the estimated heights and
$n_{coef}$ are the number of coefficients of the function which is
estimating $h_{est}$.

%\begin{figure}
%  \centering
%  \includegraphics[width=\columnwidth]{./pic/picm1}
%  \caption{Differences between observation and model on all ages where
%  height was measured.}
%  \label{fig:diffAgeMeasured}
%\end{figure}

The differences between the observed and the estimated height at the
ages 10, 30, 70, 100 and 150 have been calculated for each tree and
each growth function. In cases when height was not observed at a given
age this height was interpolated between the enclosing two data points
using a spline with the function {\tt splinefun} with method {\tt
 monoH.FC}.\\ 
%
The 0.05, 0.25, 0.5, 0.75 and 0.95 quantile for each function and
age-step has been calculated. It has been tested if there is a trend
of the difference of $h_{obs} - h_{est}$ over $h_{est}$ by fitting a
linear regression for each function and age-step by using the ``High
Breakdown and High Efficiency Robust Linear Regression'' {\tt lmRob}
\citep{robust2014}.

%\begin{figure}
%  \centering
%  \includegraphics[width=\columnwidth]{./pic/picm2}
%  \caption{Differences between observation and model at ages 10, 30,
%    70, 100 and 150 years.}
%  \label{fig:diffAgeGiven}
%\end{figure}

\subsection{Height development}

For each tree the height at age 10, 30, 70, 100, 150, 300 and 800
years was estimated with each function and the tree specific coefficients.
The 0.05, 0.25, 0.5, 0.75 and 0.95 quantile of those height estimates has
been calculated for each function and age-step.


\section{Results}

\subsection{Root--mean--square error}

For each tree its height at different ages is known. These heights
are also estimated by the different functions with tree specific
coefficients. For each tree the RSME between observed
and estimated heights is calculated. Those RSME have
been split into seven groups ranging from low to high deviation which
contain approximately the same count of trees. The number of trees
which have been classified into those groups are shown in
figure~\ref{fig:sdRankGutten} for data from Guttenberg and in
figure~\ref{fig:sdRankBFW} for the data from BFW for all growth
functions. In addition the median of the RSME has been
calculated and is given in the figures after the function name.\\
%
For the Guttenberg data-set many functions show a RSME
below $\pm 0.40$\,m and below $\pm 0.50$\,m for the BFW trees.
The group of the \emph{HyperLog} function, which uses five
coefficients, can describe the growth pattern of the trees very
accurately. Also \emph{Ln}, with only four coefficients can be found in
the middle of this best performing group. This group follows the group of
functions with four coefficients, where \emph{Hoss1E1},
\emph{Yoshida}, \emph{Sloboda}, \emph{Atan}, \emph{Tanh},
\emph{Levakovic159}, \emph{BackmanE1}, \emph{KorsunE1} and
\emph{Todorovic} are on the upper ranks. From the functions having
only three parameters \emph{Ho{\ss}feld1}, \emph{Ho{\ss}feld4}, \emph{Korsun},
\emph{Levakovic88}, \emph{Thomasius}, \emph{Backman} and
\emph{Mitscherlich} show
good results. The functions after \emph{WeberE1} for the Guttenberg
and those after \emph{Korf} for the BFW data-set show not so good
results.\\
%
By keeping the coefficient $c_1$ constant in the \emph{Ln} function
its RSME is increasing. For the BFW data-set it is better than the
best three parameter function and for the Guttenberg data-set only
Backman was slightly better. When keeping $c_3$ constant it is the
best three parameter function for the Guttenberg data-set but for the
BFW data it shows similar behaviour like the other three parameter
functions. When keeping $c_1$ and $c_3$ constant its RSME again is
increasing but it is better than \emph{Strand}, the best two parameter
function. Using a linear dependency of $c_3$ on $c_2$ shows only
marginal improves of the result.

\begin{figure*}[htbp]
  \begin{minipage}{.49\linewidth}
    \centering
    \includegraphics[trim=0 2 0 20,clip,width=\columnwidth]{../pic/sdRankGutten}
    \caption{Distribution of the root--mean--square error for single trees in meter for the data from Guttenberg.}
    \label{fig:sdRankGutten}
\end{minipage}
  \begin{minipage}{.49\linewidth}
    \centering
    \includegraphics[trim=0 2 0 20,clip,width=\columnwidth]{../pic/sdRankBFW}
    \caption{Distribution of the root--mean--square error single trees in meter for the data from BFW.}
    \label{fig:sdRankBFW}
  \end{minipage}
\end{figure*}

%\FloatBarrier

\subsection{Height difference}

For the ages 10, 30, 70, 100 and 150 the differences between
observation and model are calculated. Their 0.05, 0.25, 0.5, 0.75 and
0.95 quantile of all trees for the specific growth function are
calculated and shown in figure~\ref{fig:hdiffGutten} for the
Guttenberg and in figure~\ref{fig:hdiffBFW} for the BFW trees. Many
functions show deviations lower than 0.3\,m for 50\,\% of the trees
and in the range of 0.6\,m for 90\,\% of the trees.\\
%
The group of \emph{hyperlog} functions show the smallest differences
followed by \emph{Ln}. This group is followed by the functions of
\emph{Hoss1E1}, \emph{Sloboda}, \emph{Grosenbaugh},
\emph{Levakovic159}, \emph{KorsunE1}, \emph{Yoshida}, \emph{Atan},
\emph{Tanh} and \emph{Todorovic}. From the three parameter functions
\emph{Hossfeld1}, \emph{Backman}, \emph{Thomasius} and
\emph{Levakovic88} showed the best results.\\
%
The height at age~150 shows a tendency to be underestimated from
nearly all functions for the Guttenberg trees and to be overestimated
for the BFW
trees. For the BFW data-set a tendency of overestimation also for the
age classes 100~years can be seen. The height at the age of 10~years is for both
data-sets underestimated by many functions. An exception provides the
group of the \emph{hyperlog} functions which make bias-free estimates
also for the height at age 10.\\
%
Keeping in \emph{Ln} the coefficients $c_1$ or $c_3$ or both constant
the differences between observed and estimated heights increases. The
tendency to underestimate the height at age~100 and 150 for Guttenberg
trees disappear. For the Guttenberg trees the form with constant $c_1$
and $c_3$ show only for the age 30 years a tendency of
underestimation, for all other ages the estimates show a smaller bias
but larger variation than the function without constant coefficients.

\begin{figure*}[htbp]
  \begin{minipage}{.49\linewidth}
    \centering
    \includegraphics[width=\columnwidth]{../pic/hdiffGutten}
    \caption{Height difference Guttenberg}
    \label{fig:hdiffGutten}
  \end{minipage}
  \begin{minipage}{.49\linewidth}
    \centering
    \includegraphics[width=\columnwidth]{../pic/hdiffBFW}
    \caption{Height difference BFW}
    \label{fig:hdiffBFW}
  \end{minipage}
  \scriptsize{0.05 -- 0.95 (dotted line), 0.25 -- 0.75 (solid line)
    and 0.5 (vertical mark) quantile of $h_{observed} - h_{estimate}$ in meter
    for the ages 10 (lowest line), 30, 70, 100 and 150 (top
    line). Medians outside the shown range are indicated by dots at
    the border.}
\end{figure*}

%\FloatBarrier

The slope of the difference between $h_{observed} - h_{estimate}$
plotted over $h_{estimate}$ is shown in figure~\ref{fig:hdifTrendGut}
for Guttenberg trees and figure~\ref{fig:hdifTrendBFW} for BFW trees. Nearly
all functions show a considerable negative trend at age 10 which
means that trees whose height is estimated low are in reality bigger
and trees which are estimated to be big are in reality smaller. In
the higher age classes a slightly positive trend can be seen in the
Guttenberg data-set and a predominant negative trend in the BFW
data-set. In the oldest age class (150 years) the estimated slope is
close to zero but the RSME is bigger than in the other age
classes. Here the ranking is not so clear as in the previous
comparisons where the functions with many parameters are better than
those with fewer. Nevertheless those with 4 to 5 parameters are on the
better ranks. From those with three parameters \emph{Hossfeld1},
\emph{Backman}, \emph{Thomasius} and \emph{Levakovic88} showed good
results. Keeping in \emph{Ln} $c_1$ constant increases the slope
deviation slightly.

\begin{figure*}[htbp]
  \begin{minipage}{.49\linewidth}
    \centering
    \includegraphics[width=\columnwidth]{../pic/hdifTrendGut}
    \caption{Trend on height difference with trees from Guttenberg}
    \label{fig:hdifTrendGut}
  \end{minipage}
  \begin{minipage}{.49\linewidth}
    \centering
    \includegraphics[width=\columnwidth]{../pic/hdifTrendBFW}
    \caption{Trend on height difference with trees from BFW}
    \label{fig:hdifTrendBFW}
  \end{minipage}
  \scriptsize{Slope of $h_{observed} - h_{estimate}$  in meter plotted over
    $h_{estimate}$ for the ages 10 (lowest line), 30, 70, 100 and 150
    (top line). The small horizontal line indicates the slope
    estimation and the solid horizontal line range of its
    root--mean--square error.}
\end{figure*}

%\FloatBarrier

\subsection{Height development}

In figure~\ref{fig:hDevGut} the 0.05, 0.25, 0.5, 0.75 and 0.95 quantile of
the estimated height for Guttenberg trees for ages 10, 30, 70,
100, 150, 300 and 800 years and in figure~\ref{fig:hDevBFW} those for
BFW trees are shown. The figures are sorted by the median height
increment between 150 and 800 years. It could be seen that for the
ages up to 150 years the heights are similar and differences between
the functions compared to the observations could already be seen in
figures~\ref{fig:hdiffGutten} and \ref{fig:hdiffBFW}. \emph{HyperLog},
\emph{Ln} and \emph{Korf} still show some height increment at the
higher ages. The estimated heights are for those functions still in a
reasonable range of 40\,m to 80\,m. On the lower end is the
\emph{Logit} function which shows never negative increments. Negative
increments have been produced by the functions \emph{Gram},
\emph{Hoss1E1}, \emph{Hossfeld3}, \emph{Korsun}, \emph{KorsunE1},
\emph{KorsunE2}, \emph{LogitE2}, \emph{LogitE3}, \emph{Parabola},
\emph{Poly3}, \emph{Poly4}, \emph{PolyAllo}, \emph{Siven},
\emph{TerazakiE1} and \emph{TerazakiE2}. The estimated height
increments between 150 and 300 and also between 300 and 800 years are
typical in the range of 0\,m to 10\,m which is approximately an annual
height increment of 5\,cm/year and 1\,cm/year respectively. Keeping in
\emph{Ln} $c_1$ constant shows for the used data a slightly more
bend, and keeping $c_3$ constant a slightly more straightened height
development compared the the unrestricted function.

\begin{figure*}[htbp]
  \begin{minipage}{.49\linewidth}
    \centering
    \includegraphics[width=\columnwidth]{../pic/hDevGut}
    \caption{Height for ages 10, 30, 70, 100, 150, 300 and
      800 with Guttenberg data.}
    \label{fig:hDevGut}
  \end{minipage}
  \begin{minipage}{.49\linewidth}
    \centering
    \includegraphics[width=\columnwidth]{../pic/hDevBFW}
    \caption{Height for ages 10, 30, 70, 100, 150, 300 and
      800 with BFW data.}
    \label{fig:hDevBFW}
  \end{minipage}
  \scriptsize{0.05 -- 0.95 (dotted line), 0.25 -- 0.75 (solid line)
    and 0.5 (vertical mark) quantile of $h_{estimate}$ in meter
    for the ages 10 (lowest line), 30, 70, 100, 150, 300 and 800 (top
    line). Medians outside the shown range are indicated by dots at
    the border.}
\end{figure*}

%\FloatBarrier


\section{Discussion}

After comparing several functions it is not possible to select
\emph{the best} one. Especially those with more than three coefficients
are flexible and show very similar shapes. For functions with less
than three parameters their coefficients are easily estimated but they
are not so flexible and show larger deviation from the observed growth
pattern. Some functions with three parameters (\emph{Hossfeld1},
\emph{Backman}, \emph{Thomasius}, \emph{Levakovic88}) show acceptable
fits. Functions with four parameters show good results. The tested
functions with five parameters show only marginal
improvements. Typically one of the functions with four parameters should
be used. The final decision depends on whether the parameter
estimation converges, the parameters can be interpreted directly
(e.g.\ height at a specific age) or the function can be transformed to
show either height depending on age or estimate age as a function of
height.

Sometimes from growth functions, which describe height over age, the
first derivation can be built or functions are directly formulated as
differential equations which describe the slope of the increment at a
given age or tree size. This slope is not identical with the annual
increment. It seems admissible to use those differential functions to
estimate differences instead of differentials. But then it is
important to define for which time span this increment is estimated.
If the time span of the observed increments is a multiple of the
function time span, the function needs to be used recursively. Only
for linear functions the increments can be adjusted with a
multiplication of (time span function) / (time span observation).
However, this simple increment adjustment can be used to find starting
parameters either for the recursive function call or the integral
function form.

Tests, whether differences between the functions are significant or
not, have not not been made, as the differences between the functions
should have practical relevance and not statistical significance. If
the difference between two height growth functions is statistically
significant, but the difference is in a range of practical
irrelevance, I would recommend to choose the simpler function. Also
the criterion if the difference is significant or not depends beside
the selected significance level on the attributes of subsample and
its size. The influence of different samples on the function ranking
by the applied methods is shown by using the two different height
growth data sets. There is no need that a function showing good
results on the used data sets is also good on other data sets. To find
the functions which can follow the growth pattern in a good way on a
different data set, all functions need to be tested. But it can
be expected that functions with low deviations on the two
data sets used here show also good behaviour on comparable data sets.

For the function $h = c_0 \cdot \ln^{c_1} (1 + c_2\cdot t^{c_3})$ it
was tried to keep up to two coefficients constant for each
data-set. This reduces the flexibility of the function. Keeping $c_1$
constant shows for both data-sets good results which are close to the
best or better than the best three parameter function. Keeping $c_3$
constant shows good results for the Guttenberg trees. When holding
$c_1$ and $c_3$ constant the deviation comes close to the range from
acceptable to unacceptable but is still better than any of the tested
two parameter functions. It could be expected, that keeping
coefficients constant, but estimating it optimal for the used
data-set, should result in better results than functions which do not
have, or set some coefficients to a data independent fixed value.

If the linear intercorrelation between the coefficients makes it
difficult, for the used algorithm, to find optimal parameters the
formulations
$h = c_0\cdot \ln^\frac{c_1}{c_2}(1 + e^{c_2} \cdot t^{c_2 \cdot c_3})$,
$h = c_0 \cdot \frac{\ln^{c_1} (1 +  c_2\cdot t^{c_3})} {\ln^{c_1} (1 + c_2\cdot 100^{c_3})}$
or
$h = c_0\cdot \frac{\ln^\frac{c_1}{c_2}(1 + e^{c_2} \cdot t^{c_2 \cdot c_3})} {\ln^\frac{c_1}{c_2}(1 + e^{c_2} \cdot 100^{c_2 \cdot c_3})}$
could be tried, which have reduced the coefficient
intercorrelation for the used data-sets. Those are only alternative
formulations of the \emph{Ln} function which might help the solver,
but will not change the shape of the resulting curve. The last two
forms make $c_0$ equal with the height at age 100.

\section{Conclusions}

From the examined 55 functions
$h  = c_0 \cdot \ln^{c_1} (1 + c_2\cdot t^{c_3})$
showed good behaviour and might be one which could be recommended.
Only for the young
ages, around 10~years their height is underestimated. The group of
\emph{Hyper-Logistic} functions dose not show this weakness. But they
need an additional parameter and can directly show only the increments
over height. The differences to functions like \emph{Sloboda},
\emph{Todorovic}, \emph{Levakovic159}, \emph{Grosenbaugh}, \emph{Tanh}
and \emph{Atan} are marginal. From the three parameter functions
\emph{Hossfeld1}, \emph{Backman}, \emph{Thomasius} and
\emph{Levakovic88} showed good results.\\[1em]

%\begin{acknowledgements}
Many thanks to Thomas Ledermann and Markus Neumann for their
constructive comments, Margareta Khorchidi for proof reading and
Hubert Sterba and Hubert Hasenauer for reviewing the manuscript.
%\end{acknowledgements}

%\subsection{Funding}

\bibliography{lit}
\bibliographystyle{plainnat}

\end{document}
